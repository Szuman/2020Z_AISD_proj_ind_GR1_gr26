\documentclass{article}
\usepackage[utf8]{inputenc}
\usepackage{tcolorbox}

\title{Specyfikacja implementacyjna}
\author{Piotr Szumański}
\date{Listopad 2020}

\begin{document}

\maketitle

\section{Informacje ogólne}
\begin{enumerate}
    \item Nazwa programu
    \\ Projekt indywidualny 2020
    \item Poruszany problem
    \\ Minimalizacja kosztów sprzedaży szczepionek przy jednoczesnym zapewnieniu dostaw do wszystkich aptek.
\end{enumerate}

\section{Wstęp}
W projekcie mamy doczynienia z jednym z rodzaju problemu marszrutyzacji. W związku z tym znalezienie idealnego rozwiązania nie jest takie proste i oczywiste. Dlatego w moim projekcie zastosuje jeden z algorytmów heurystycznych, które jest rozwiązaniem problemu dla niektórych kombinacji danych.

\section{Opis działania programu}
    \begin{enumerate}
        \item Program prosi użytkownika o podanie ścieżki do pliku z danymi
        \item Pobiera dane z podanego pliku za pomocą klasy FileReader i wpisuje je do odpowiednich list w klasie Result w zależności od rodzaju danych (lista producentów, aptek czy połączeń)
        \item Po pobraniu danych do list, sortuje listę połączeń po cenie (od najtańszego) za pomocą Collection.sort
        \item Następnie iteruje po tej liście połączeń i najpierw pobiera dane producenta i apteki, których id jest w rozpatrywanym połączeniu
        \item Wybiera najmniejszą z trzech wartości 
        \begin{itemize}
            \item dziennej produkcji
            \item dziennego zapotrzebowania
            \item maksymalnej liczby dostarczanych szczepionek
        \end{itemize}
        \item Wstawiamy odpowiednie wartości do pliku wynikowego (wynik.txt), a mianowicie:
            \begin{tcolorbox}
                 Producent X $->$ Apteka Y [Koszt = (najm. watość) * (cena) = wynik]
            \end{tcolorbox}
        \item Dodajemy wynik do sumy opłat całkowitych
        \item Użytą najmniejszą wartość odejmujemy od dziennej produkcji i dziennego zapotrzebowania
        \item Bierzemy następne połączenie
        \item Po iteracji wstawiamy sumę opłat do pliku z wynikami jako: 
            \begin{tcolorbox}
                Opłaty całkowite: "suma" zł
            \end{tcolorbox}
        \\ W przypadku błednych danych w dokumencie czy źle napisanej ścieżki do pliku program natychmiast powiadomi użytkownika o błedzie
    \end{enumerate}

\section{Opis obiektów}
W tych obiektach występują odpowiednie konstruktory oraz gettery dla wszystkich podanych zmiennych.
    \item Manufacturer - obiekt producent
        \begin{itemize}
        \item int id - id producenta
        \item String name - nazwa producenta
        \iten int dailyProduction - dzienna produkcja (do tej zmiennej używany jest dodatkowo setter)
    \end{itemize}
    \item Pharmacy - obiekt apteka
        \begin{itemize}
            \item int id - id apteki
            \item String name - nazwa apteki
            \item int dailyRequire - dzienne zapotrzebowanie (do tej zmiennej używany jest dodatkowo setter)
        \end{itemize}
    \item Connection - obiekt połączenie, które implementuje interfejs Comparator$<$Connection$>$
        \begin{itemize}
            \item int idM - id producenta
            \item int idP - id apteki
            \item int maxDaily - dzienna maksymalna liczba dostaczanych szczepionek
            \item double price - cena
            \item compareTo() - metoda nadpisana od Comparatora, napisana by porównywała połączenia po cenie, a w przydadku remisu porównywała po max. liczbie dostarczeń (odwrotnie jak cena)
        \end{itemize}
    
\section{Opis klas i metod}
\begin{enumerate}
    \item Main - klasa z główną metodą inicjalizującą działanie programu
    \begin{itemize}
        \item main() - prosi o wprowadzenie ścieżki pliku oraz wywołuje klasę Result 
    \end{itemize}
    \item Result - oblicza najlepszą możliwość na podstawie dostarczonej ścieżki do pliku i wysyła ją do pliku wynik.txt
    \begin{itemize}
        \item void calculate() - metoda z algorytmem 
        \item Manufacturer findManufacturer(int idM) - zwraca producenta na podstawie id producenta z danego połączenia
        \item Pharmacy findPharmacy(int idP) - zwraca aptekę na podstawie id apteki z listy połączeń
        \item int minValueOfThree(int a, int b, int c) - zwraca najmniejszą z trzech podanych wartości
        \item void writeAnwser(String line) - wysyła podany ciąg znaków do pliku wynik.txt
    \end{itemize}
    \item FileReader - wczytuje odpowiednie dane do danych list: 
    \begin{itemize}
        \item List$<$Manufacturer$>$ manufacturers - lista producentów          \item List$<$Pharmacy$>$ pharmacies - lista aptek
        \item List$<$Connection$>$ connections - lista połączeń
    \end{itemize}
    \\Dla tych list zostaną dodane gettery, aby klasa Result miała do nich dostęp.
     \begin{itemize}
        \item loadData(String filePath) - metoda do wczytywania danych z pliku na podstawie podanej ścieżki, wywoływana w klasie Result
        \item void addManufacturer(String line) - dodaje nowo wytworzonego producenta do listy producentów, pobiera dane z danego ciągu znaków
        \item void addPharmacy(String line) - dodaje nowo wytworzoną aptekę do listy aptek, pobiera dane z danego ciągu znaków
        \item void addConnection(String line) - dodaje nowo wytworzone połączenie do listy połączeń, pobiera dane z danego ciągu znaków
    \end{itemize}
\end{enumerate}

\section{Zakończenie}
W przydadku powodzenia się tego algorytmu możliwa jest opcja dodania jeszcze jednego algorytmu dla zwiększenia prawdopodobienstwa znalezienia poprawnej konfiguracji oraz dodania testów dla każdego z tych algorytmów.

\end{document}
